% Resources for the AASTeX:
%   https://journals.aas.org/aastex-package-for-manuscript-preparation
% Author guide:
%   https://journals.aas.org/aastexguide/

\documentclass[modern]{aastex7}

% Package imports go here.

% Local commands go here.
\newcommand{\docRef}{TESTN-000}
\newcommand{\docUpstreamLocation}{\url{https://github.com/lsst-dm/testn-000}}

%%%%%%%%%%%%%%%%%%%%%%%%%%%%%%%%%%%%%%%%%%%%%%%%%%%%%%%%%%%%%%%%%%%%%%%%%%%%%%%%
%%
%% The following section outlines numerous optional output that
%% can be displayed in the front matter or as running meta-data.
%%
%% Running header information. A short title on odd pages and 
%% short author list on even pages. Note that this
%% information may be modified in production.
%%\shorttitle{AASTeX v7 Sample article}
%%\shortauthors{The Terra Mater collaboration}
%%
%% Include dates for submitted, revised, and accepted.
%%\received{February 1, 2025}
%%\revised{March 1, 2025}
%%\accepted{\today}
%%
%% Indicate AAS Journal the manuscript was submitted to.
%% Note that this command adds "Submitted to " the argument.
%%\submitjournal{PSJ}
%%
%% You can add a light gray and diagonal water-mark to the first page 
%% with this command:
%% \watermark{text}
%% where "text", e.g. DRAFT, is the text to appear.  If the text is 
%% long you can control the water-mark size with:
%% \setwatermarkfontsize{dimension}
%% where dimension is any recognized LaTeX dimension, e.g. pt, in, etc.
%%%%%%%%%%%%%%%%%%%%%%%%%%%%%%%%%%%%%%%%%%%%%%%%%%%%%%%%%%%%%%%%%%%%%%%%%%%%%%%%
%%
%% Use this command to indicate a subdirectory where figures are located.
%%\graphicspath{{./}{figures/}}


\begin{document}
%% DO NOT EDIT THIS FILE. IT IS GENERATED FROM db2authors.py"
%% Regenerate using:
%%    python $LSST_TEXMF_DIR/bin/db2authors.py > authors.tex


\author[1]{William~O'Mullane}
\affil[1]{Rubin Observatory Project Office, 950 N.\ Cherry Ave., Tucson, AZ  85719, USA}


\date{\today}
\title{Document Title}

% This can write metadata into the PDF.
% Update keywords and author information as necessary.
\hypersetup{
    pdftitle={Document Title},
    pdfauthor={omullanew},
    pdfkeywords={}
}



\begin{abstract}

{{ cookiecutter.abstract }}

\end{abstract}



%%Graphical abstract

%\begin{graphicalabstract}

%\includegraphics{grabs}

%\end{graphicalabstract}



%% Keywords should appear after the \end{abstract} command.
%% The AAS Journals now uses Unified Astronomy Thesaurus concepts:
%% https://astrothesaurus.org
%% You can use the \uat command to link your UAT concepts back its source.
%% \keywords{\uat{Galaxies}{573} --- \uat{Cosmology}{343}}

%% Main content as individual inputs.
\section{Introduction}

{\bf Put your paper here }
\vskip 0.4in



This is the LSST overview paper: \cite{2008arXiv0805.2366I}.




%% Modify acknowledgments as needed.
\begin{acknowledgments}
This material is based upon work supported in part by the National Science Foundation through Cooperative Agreements AST-1258333 and AST-2241526 and Cooperative Support Agreements AST-1202910 and AST-2211468 managed by the Association of Universities for Research in Astronomy (AURA), and the Department of Energy under Contract No.\ DE-AC02-76SF00515 with the SLAC National Accelerator Laboratory managed by Stanford University.
Additional Rubin Observatory funding comes from private donations, grants to universities, and in-kind support from LSST-DA Institutional Members.
\end{acknowledgments}

%% To help institutions obtain information on the effectiveness of their
%% telescopes the AAS Journals has created a group of keywords for telescope
%% facilities.
%
%% Following the acknowledgments section, use the following syntax and the
%% \facility{} or \facilities{} macros to list the keywords of facilities used
%% in the research for the paper.  Each keyword is check against the master
%% list during copy editing.  Individual instruments can be provided in
%% parentheses, after the keyword, but they are not verified.

\vspace{5mm}
\facility{Rubin}

%% Similar to \facility{}, there is the optional \software command to allow
%% authors a place to specify which programs were used during the creation of
%% the manuscript. Authors should list each code and include either a
%% citation or url to the code inside ()s when available.
\software{LSST Science Pipelines}

%% Appendix material should be preceded with a single \appendix command.
%% There should be a \section command for each appendix. Mark appendix
%% subsections with the same markup you use in the main body of the paper.

%% Each Appendix (indicated with \section) will be lettered A, B, C, etc.
%% The equation counter will reset when it encounters the \appendix
%% command and will number appendix equations (A1), (A2), etc. The
%% Figure and Table counter will not reset.

%% \appendix


% Include all the relevant bib files so that references can be found from
% lsst-texmf.
% https://lsst-texmf.lsst.io/lsstdoc.html#bibliographies
\bibliographystyle{aasjournal}
\bibliography{local,lsst,lsst-dm,refs_ads,refs,books}

\end{document}
