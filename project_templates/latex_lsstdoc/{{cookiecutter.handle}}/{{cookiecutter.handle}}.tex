\documentclass[{{ cookiecutter.lsstdoc_org }},toc]{lsstdoc}
% lsstdoc documentation: https://lsst-texmf.lsst.io/lsstdoc.html

% Generated by Makefile
% GENERATED FILE -- edit this in the Makefile
\newcommand{\lsstDocType}{TESTN}
\newcommand{\lsstDocNum}{000}
\newcommand{\vcsRevision}{ebc6a8b-dirty}
\newcommand{\vcsDate}{2022-07-04}


% Package imports go here.

% Local commands go here.

% If you want glossaries, uncomment:
% \input{aglossary.tex}
% \makeglossaries

\title{ {{- cookiecutter.title -}} }
% \setDocSubtitle{Optional subtitle}

%% DO NOT EDIT THIS FILE. IT IS GENERATED FROM db2authors.py"
%% Regenerate using:
%%    python $LSST_TEXMF_DIR/bin/db2authors.py > authors.tex


\author[1]{William~O'Mullane}
\affil[1]{Rubin Observatory Project Office, 950 N.\ Cherry Ave., Tucson, AZ  85719, USA}



\setDocRef{ {{- cookiecutter.handle -}} }
\setDocUpstreamLocation{\url{https://github.com/{{ cookiecutter.github_org }}/{{ cookiecutter.handle -}} }}
\date{\vcsDate}
% \setDocCurator{The Curator of this Document}


\setDocAbstract{%
Add abstract text.
}

% Revision history.
% Order: oldest first.
% Fields: VERSION, DATE, DESCRIPTION, OWNER NAME.
% See LPM-51 for version number policy.

\setDocChangeRecord{%
  \addtohist{1}{ {{- cookiecutter.date -}} }{Unreleased.}{ {{- cookiecutter.first_author_family -}} }
}

\begin{document}

\maketitle

% ADD CONTENT HERE
% You can also use the \input command to include several content files.

\appendix

\section{Acknowledgements}

This material is based upon work supported in part by the National Science Foundation through Cooperative Agreements AST-1258333 and AST-2241526 and Cooperative Support Agreements AST-1202910 and AST-2211468 managed by the Association of Universities for Research in Astronomy (AURA), and the Department of Energy under Contract No.\ DE-AC02-76SF00515 with the SLAC National Accelerator Laboratory managed by Stanford University.
Additional Rubin Observatory funding comes from private donations, grants to universities, and in-kind support from LSST-DA Institutional Members.

% Include all the relevant bib files.
% https://lsst-texmf.lsst.io/lsstdoc.html#bibliographies
\section{References} \label{sec:bib}
\renewcommand{\refname}{} % Suppress default Bibliography section
\bibliography{local,lsst,lsst-dm,refs_ads,refs,books}

% Make sure lsst-texmf/bin/generateAcronyms.py is in your path
\section{Acronyms} \label{sec:acronyms}
\addtocounter{table}{-1}
\begin{longtable}{p{0.145\textwidth}p{0.8\textwidth}}\hline
\textbf{Acronym} & \textbf{Description}  \\\hline

AST & NSF Division of Astronomical Sciences \\\hline
AURA & Association of Universities for Research in Astronomy \\\hline
DAQ & Data Acquisition System \\\hline
DE & dark energy \\\hline
DM & Data Management \\\hline
EPO & Education and Public Outreach \\\hline
LSST & Legacy Survey of Space and Time (formerly Large Synoptic Survey Telescope) \\\hline
SLAC & SLAC National Accelerator Laboratory \\\hline
\end{longtable}

% If you want glossary uncomment below and comment out the two lines above.
% \printglossaries

\end{document}
