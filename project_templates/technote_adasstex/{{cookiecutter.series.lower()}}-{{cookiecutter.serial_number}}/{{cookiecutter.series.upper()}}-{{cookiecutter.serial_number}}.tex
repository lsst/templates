
\documentclass[11pt,twoside]{article}

% Do NOT use ANY packages other than asp2014.
\usepackage{asp2014}
%if you add acronyms - but asp say no other imports
%\usepackage{longtable}

\aspSuppressVolSlug
\resetcounters

% References must all use BibTeX entries in a .bibfile.
% References must be cited in the text using \citet{} or \citep{}.
% Do not use \cite{}.
% See ManuscriptInstructions.pdf for more details
\bibliographystyle{asp2014}
\def\procspie{Proc.\ SPIE} % Proceedings of the SPIE


% lsstdoc documentation: https://lsst-texmf.lsst.io/lsstdoc.html
% GENERATED FILE -- edit this in the Makefile
\newcommand{\lsstDocType}{TESTN}
\newcommand{\lsstDocNum}{000}
\newcommand{\vcsRevision}{ebc6a8b-dirty}
\newcommand{\vcsDate}{2022-07-04}


% Package imports go here.

% Local commands go here.

% See ASPmanual2010.pdf 2.1.4  and ManuscriptInstructions.pdf for more details
%\markboth{auth}{short title}


\newcommand{\docRef}{ {{- cookiecutter.series.upper() -}}-{{- cookiecutter.serial_number -}} }
\newcommand{\docUpstreamLocation}{\url{https://github.com/{{ cookiecutter.github_org }}/{{ cookiecutter.series.lower()}}-{{ cookiecutter.serial_number -}} }}


\begin{document}
%% DO NOT EDIT THIS FILE. IT IS GENERATED FROM db2authors.py"
%% Regenerate using:
%%    python $LSST_TEXMF_DIR/bin/db2authors.py > authors.tex


\author[1]{William~O'Mullane}
\affil[1]{Rubin Observatory Project Office, 950 N.\ Cherry Ave., Tucson, AZ  85719, USA}


\date{\today}
\title{ {{- cookiecutter.title -}} }

% This can write metadata into the PDF.
% Update keywords and author information as necessary.
\hypersetup{
    pdftitle={ {{- cookiecutter.title -}} },
    pdfauthor={ {{- cookiecutter.author_id -}} },
    pdfkeywords={}
}



\begin{abstract}

{{ cookiecutter.abstract }}

\end{abstract}



%%Graphical abstract

%\begin{graphicalabstract}

%\includegraphics{grabs}

%\end{graphicalabstract}



\section{Introduction}

{\bf Put your paper here }
\vskip 0.4in



This is the LSST overview paper: \cite{2008arXiv0805.2366I}.



\appendix
% Include all the relevant bib files.
% https://lsst-texmf.lsst.io/lsstdoc.html#bibliographies
\section{References} \label{sec:bib}
\bibliography{local,lsst,lsst-dm,refs_ads,refs,books}

%Usually no space for acronyms in adass 4 pager
% Make sure lsst-texmf/bin/generateAcronyms.py is in your path
%\section{Acronyms} \label{sec:acronyms}
%\addtocounter{table}{-1}
\begin{longtable}{p{0.145\textwidth}p{0.8\textwidth}}\hline
\textbf{Acronym} & \textbf{Description}  \\\hline

AST & NSF Division of Astronomical Sciences \\\hline
AURA & Association of Universities for Research in Astronomy \\\hline
DAQ & Data Acquisition System \\\hline
DE & dark energy \\\hline
DM & Data Management \\\hline
EPO & Education and Public Outreach \\\hline
LSST & Legacy Survey of Space and Time (formerly Large Synoptic Survey Telescope) \\\hline
SLAC & SLAC National Accelerator Laboratory \\\hline
\end{longtable}

\noindent {\tiny This material or work is supported in part by the National Science Foundation through Cooperative Agreement AST-1258333 and Cooperative Support Agreement AST1836783 managed by the Association of Universities for Research in Astronomy (AURA), and the Department of Energy under Contract No. DE-AC02-76SF00515 with the SLAC National Accelerator Laboratory managed by Stanford University.
}

\end{document}
