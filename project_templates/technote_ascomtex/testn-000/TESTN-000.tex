%%
%% Copyright 2007-2024 Elsevier Ltd
%%
%% This file is part of the 'Elsarticle Bundle'.
%% ---------------------------------------------
%%
%% It may be distributed under the conditions of the LaTeX Project Public
%% License, either version 1.3 of this license or (at your option) any
%% later version.  The latest version of this license is in
%%    http://www.latex-project.org/lppl.txt
%% and version 1.3 or later is part of all distributions of LaTeX
%% version 1999/12/01 or later.
%%
%% The list of all files belonging to the 'Elsarticle Bundle' is
%% given in the file `manifest.txt'.
%%
%% Template article for Elsevier's document class `elsarticle'
%% with numbered style bibliographic references
%% SP 2008/03/01
%% $Id: elsarticle-template-num.tex 249 2024-04-06 10:51:24Z rishi $
%%
\documentclass[preprint,12pt]{elsarticle}

%% Use the option review to obtain double line spacing
%% \documentclass[authoryear,preprint,review,12pt]{elsarticle}

%% Use the options 1p,twocolumn; 3p; 3p,twocolumn; 5p; or 5p,twocolumn
%% for a journal layout:
%% \documentclass[final,1p,times]{elsarticle}
%% \documentclass[final,1p,times,twocolumn]{elsarticle}
%% \documentclass[final,3p,times]{elsarticle}
%% \documentclass[final,3p,times,twocolumn]{elsarticle}
%% \documentclass[final,5p,times]{elsarticle}
%% \documentclass[final,5p,times,twocolumn]{elsarticle}

%% For including figures, graphicx.sty has been loaded in
%% elsarticle.cls. If you prefer to use the old commands
%% please give \usepackage{epsfig}

%% The amssymb package provides various useful mathematical symbols
\usepackage{amssymb}
%% The amsmath package provides various useful equation environments.
\usepackage{amsmath}
\usepackage{hyperref}
%% The amsthm package provides extended theorem environments
%% \usepackage{amsthm}

%% The lineno packages adds line numbers. Start line numbering with
%% \begin{linenumbers}, end it with \end{linenumbers}. Or switch it on
%% for the whole article with \linenumbers.
%% \usepackage{lineno}

\journal{Astronomy and Computing}

% Local commands go here.
\newcommand{\docRef}{TESTN-000}
\newcommand{\docUpstreamLocation}{\url{https://github.com/lsst-dm/testn-000}}

\providecommand{\secref}[1]{Section~\ref{#1}}
\providecommand{\appref}[1]{Appendix~\ref{#1}}
\providecommand{\tabref}[1]{Table~\ref{#1}}
\providecommand{\figref}[1]{Figure~\ref{#1}}
\providecommand{\eqnref}[1]{Eq.~\ref{#1}}
\providecommand{\recref}[1]{REC-\ref{#1}}
\def\VRO{Vera C. Rubin Observatory~}
\def\RO{Rubin Observatory~}
\def\aaps{A\&AS}           % Astronomy and Astrophysics Suplement
\def\aap{A\&A}             % Astronomy and Astrophysics
\def\ssr{Space~Sci.~Rev.}  % Space Science Reviews
\def\apj{ApJ}              % Astrophysical Journal
\def\apjs{ApJS}            % Astrophysical Journal Supplement
\def\aj{AJ}                % Astronomical Journal
\def\mnras{MNRAS}          % Monthly Notices of the RAS
\def\araa{ARA\&A}          % Annual Review of Astron and Astrophys
\def\nat{Nature}           % Nature
\def\apjl{ApJ}             % Astrophysical Journal, Letters
\def\icarus{Icarus}        % Icarus
\def\prd{Phys.~Rev.~D}     % Physical Review D
\def\physrep{Phys.~Rep.}   % Physics Reports
\def\pasp{PASP}            % Publications of the Astronomical Society of the Pacific
\def\procspie{Proc.\ SPIE} % Proceedings of the SPIE
\newcommand{\pasa}{PASA}   % Publications of the Astronomical Society of Australia
\newcommand{\ao}{Appl.~Opt.}  % Applied Optics
\def\pasj{PASJ}            % Publications of the Astronomical Society of Japan

\begin{document}

\begin{frontmatter}

%% Title, authors and addresses

%% use the tnoteref command within \title for footnotes;
%% use the tnotetext command for theassociated footnote;
%% use the fnref command within \author or \affiliation for footnotes;
%% use the fntext command for theassociated footnote;
%% use the corref command within \author for corresponding author footnotes;
%% use the cortext command for theassociated footnote;
%% use the ead command for the email address,
%% and the form \ead[url] for the home page:
%% \title{Title\tnoteref{label1}}
%% \tnotetext[label1]{}
%% \author{Name\corref{cor1}\fnref{label2}}
%% \ead{email address}
%% \ead[url]{home page}
%% \fntext[label2]{}
%% \cortext[cor1]{}
%% \affiliation{organization={},
%%             addressline={},
%%             city={},
%%             postcode={},
%%             state={},
%%             country={}}
%% \fntext[label3]{}
%% DO NOT EDIT THIS FILE. IT IS GENERATED FROM db2authors.py"
%% Regenerate using:
%%    python $LSST_TEXMF_DIR/bin/db2authors.py > authors.tex


\author[1]{William~O'Mullane}
\affil[1]{Rubin Observatory Project Office, 950 N.\ Cherry Ave., Tucson, AZ  85719, USA}


\date{\today}
\title{Document Title}

% This can write metadata into the PDF.
% Update keywords and author information as necessary.
\hypersetup{
    pdftitle={Document Title},
    pdfauthor={omullanew},
    pdfkeywords={}
}



\begin{abstract}

{{ cookiecutter.abstract }}

\end{abstract}



%%Graphical abstract

%\begin{graphicalabstract}

%\includegraphics{grabs}

%\end{graphicalabstract}



\begin{keyword}
Vera C. Rubin Observatory
%% keywords here, in the form: keyword \sep keyword

%% PACS codes here, in the form: \PACS code \sep code

%% MSC codes here, in the form: \MSC code \sep code
%% or \MSC[2008] code \sep code (2000 is the default)

\end{keyword}

\end{frontmatter}


\section{Introduction}

{\bf Put your paper here }
\vskip 0.4in



This is the LSST overview paper: \cite{2008arXiv0805.2366I}.



%% For citations use:
%%       \cite{<label>} ==> [1]



%% Modify acknowldgments as needed.
\textbf{Acknowledgments}\\
This material is based upon work supported in part by the National Science Foundation through Cooperative Agreement AST-1258333 and Cooperative Support Agreement AST-1202910 managed by the Association of Universities for Research in Astronomy (AURA), and the Department of Energy under Contract No. DE-AC02-76SF00515 with the SLAC National Accelerator Laboratory managed by Stanford University.
Additional Rubin Observatory funding comes from private donations, grants to universities, and in-kind support from LSSTC Institutional Members.


%% \appendix


% Include all the relevant bib files so that references can be found from
% lsst-texmf.
% https://lsst-texmf.lsst.io/lsstdoc.html#bibliographies
\bibliographystyle{elsarticle-num}
\bibliography{local,lsst,lsst-dm,refs_ads,refs,books}

\end{document}
