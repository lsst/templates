\documentclass[modern]{aastex62}

% lsstdoc documentation: https://lsst-texmf.lsst.io/lsstdoc.html
% GENERATED FILE -- edit this in the Makefile
\newcommand{\lsstDocType}{TESTN}
\newcommand{\lsstDocNum}{000}
\newcommand{\vcsRevision}{3aff761-dirty}
\newcommand{\vcsDate}{2024-04-09}


% Package imports go here.

% Local commands go here.

\title{ {{- cookiecutter.title -}} }


\newcommand{\docRef}{set the Reference with {$\backslash$}setDocRef}
\newcommand{\setDocRef}[1]{
   \renewcommand{\docRef}{#1}
}
\newcommand{\docRef}{ {{- cookiecutter.series.upper() -}}-{{- cookiecutter.serial_number -}} }
\setDocUpstreamLocation{\url{https://github.com/{{ cookiecutter.github_org }}/{{ cookiecutter.series.lower()}}-{{ cookiecutter.serial_number -}} }}


\begin{document}
%% DO NOT EDIT THIS FILE. IT IS GENERATED FROM db2authors.py"
%% Regenerate using:
%%    python $LSST_TEXMF_DIR/bin/db2authors.py Namespace(mode='spie', noafil=False) 

\author[1]{~William~O'Mullane}
\affil[1]{Vera C. Rubin Observatory, Avenida Juan Cisternas \#1500, La Serena, Chile}


\date{\today}


\begin{abstract}

{{ cookiecutter.abstract }}

\end{abstract}



% Create the title page.
\maketitle
\section{Introduction}

{\bf Put your paper here }
\vskip 0.4in

This is the LSST overview paper: \cite{2008arXiv0805.2366I}.



\appendix
% Include all the relevant bib files.
% https://lsst-texmf.lsst.io/lsstdoc.html#bibliographies
\section{References} \label{sec:bib}
\bibliography{local,lsst,lsst-dm,refs_ads,refs,books}

% Make sure lsst-texmf/bin/generateAcronyms.py is in your path
\section{Acronyms} \label{sec:acronyms}
\addtocounter{table}{-1}
\begin{longtable}{p{0.145\textwidth}p{0.8\textwidth}}\hline
\textbf{Acronym} & \textbf{Description}  \\\hline

DM & Data Management \\\hline
\end{longtable}


\end{document}
