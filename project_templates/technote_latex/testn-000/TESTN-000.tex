\documentclass[DM,authoryear,toc]{lsstdoc}
% lsstdoc documentation: https://lsst-texmf.lsst.io/lsstdoc.html
% GENERATED FILE -- edit this in the Makefile
\newcommand{\lsstDocType}{TESTN}
\newcommand{\lsstDocNum}{000}
\newcommand{\vcsRevision}{3aff761-dirty}
\newcommand{\vcsDate}{2024-04-09}


% Package imports go here.

% Local commands go here.

\title{Document Title}

% Optional subtitle
% \setDocSubtitle{A subtitle}

\author{%
First Last
}

\setDocRef{TESTN-000}
\setDocUpstreamLocation{\url{https://github.com/lsst-dm/testn-000}}

\date{\vcsDate}

% Optional: name of the document's curator
% \setDocCurator{The Curator of this Document}

\setDocAbstract{%
Abstract text.
}

% Change history defined here.
% Order: oldest first.
% Fields: VERSION, DATE, DESCRIPTION, OWNER NAME.
% See LPM-51 for version number policy.
\setDocChangeRecord{%
  \addtohist{1}{YYYY-MM-DD}{Unreleased.}{First Last}
}

% Frequently for a technote we do not want a title page  uncomment this to remove the title page and changelog.
%\setDocCompact{true}

\begin{document}

% Create the title page.
\maketitle

% ADD CONTENT HERE
% You can also use the \input command to include several content files.

\appendix
% Include all the relevant bib files.
% https://lsst-texmf.lsst.io/lsstdoc.html#bibliographies
\section{References} \label{sec:bib}
\bibliography{local,lsst,lsst-dm,refs_ads,refs,books}

% Make sure lsst-texmf/bin/generateAcronyms.py is in your path
\section{Acronyms} \label{sec:acronyms}
\addtocounter{table}{-1}
\begin{longtable}{p{0.145\textwidth}p{0.8\textwidth}}\hline
\textbf{Acronym} & \textbf{Description}  \\\hline

DM & Data Management \\\hline
\end{longtable}


\end{document}
