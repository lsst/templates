\documentclass[{{ cookiecutter.org }},lsstdraft,authoryear,toc]{lsstdoc}
% GENERATED FILE -- edit this in the Makefile
\newcommand{\lsstDocType}{TESTN}
\newcommand{\lsstDocNum}{000}
\newcommand{\vcsRevision}{ebc6a8b-dirty}
\newcommand{\vcsDate}{2022-07-04}


% Package imports go here.

% Local commands go here.

%If you want glossaries
%\input{aglossary.tex}
%\makeglossaries

\title{ {{- cookiecutter.title -}} }

% This can write metadata into the PDF.
% Update keywords and author information as necessary.
\hypersetup{
    pdftitle={ {{- cookiecutter.title -}} },
    pdfauthor={ {{- cookiecutter.first_author -}} },
    pdfkeywords={}
}

% Optional subtitle
% \setDocSubtitle{A subtitle}


\author{%
{{ cookiecutter.first_author }}
}

\setDocRef{ {{- cookiecutter.series.upper() -}}-{{- cookiecutter.serial_number -}} }
\setDocUpstreamLocation{\url{https://github.com/{{ cookiecutter.github_org }}/{{ cookiecutter.series.lower()}}-{{ cookiecutter.serial_number -}} }}

\date{\vcsDate}

% Optional: name of the document's curator
% \setDocCurator{The Curator of this Document}


\setDocAbstract{%
{{ cookiecutter.abstract }}
}

% Change history defined here.
% Order: oldest first.
% Fields: VERSION, DATE, DESCRIPTION, OWNER NAME.
% See LPM-51 for version number policy.

\setDocChangeRecord{%
  \addtohist{1}{YYYY-MM-DD}{Unreleased.}{ {{- cookiecutter.first_author -}} }
}


\begin{document}

% Create the title page.
\maketitle
% Frequently for a technote we do not want a title page  uncomment this to remove the title page and changelog.
% use \mkshorttitle to remove the extra pages

% ADD CONTENT HERE
% You can also use the \input command to include several content files.

\appendix
% Include all the relevant bib files.
% https://lsst-texmf.lsst.io/lsstdoc.html#bibliographies
\section{References} \label{sec:bib}
\renewcommand{\refname}{} % Suppress default Bibliography section
\bibliography{local,lsst,lsst-dm,refs_ads,refs,books}

% Make sure lsst-texmf/bin/generateAcronyms.py is in your path
\section{Acronyms} \label{sec:acronyms}
\addtocounter{table}{-1}
\begin{longtable}{p{0.145\textwidth}p{0.8\textwidth}}\hline
\textbf{Acronym} & \textbf{Description}  \\\hline

AST & NSF Division of Astronomical Sciences \\\hline
AURA & Association of Universities for Research in Astronomy \\\hline
DAQ & Data Acquisition System \\\hline
DE & dark energy \\\hline
DM & Data Management \\\hline
EPO & Education and Public Outreach \\\hline
LSST & Legacy Survey of Space and Time (formerly Large Synoptic Survey Telescope) \\\hline
SLAC & SLAC National Accelerator Laboratory \\\hline
\end{longtable}

% If you want glossary uncomment below -- comment out the two lines above
%\printglossaries





\end{document}
